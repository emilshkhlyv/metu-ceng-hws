\documentclass[10pt,a4paper, margin=1in]{article}
\usepackage{fullpage}
\usepackage{amsfonts, amsmath, pifont}
\usepackage{amsthm}
\usepackage{graphicx}
\usepackage{tikz}
\usepackage{graphicx}
\usepackage{tkz-euclide}
\usepackage{pgfplots}

\begin{filecontents}{q3a.dat}
  n   xn
 -1  1
  0  2
 
\end{filecontents}
\begin{filecontents}{q3b.dat}
  n   xn
 -1  2
  0  0
  1  1
 
\end{filecontents}
\begin{filecontents}{q3c.dat}
  n   xn
 -2  2
  -1  0
  0  1
 
\end{filecontents}
\begin{filecontents}{q3d.dat}
  n   xn
 -2  2
  -1  4
  0  1
  1 2
\end{filecontents}
\begin{filecontents}{q3e.dat}
  n   xn
  -1  4
  1 2
\end{filecontents}
\usepackage{geometry}
 \geometry{
 a4paper,
 total={210mm,297mm},
 left=10mm,
 right=10mm,
 top=10mm,
 bottom=10mm,
 }
 % Write both of your names here. Fill exxxxxxx with your ceng mail address.
 \author{
  Shikhaliyev, Emil\\
  \texttt{e2280386@ceng.metu.edu.tr}
  \and
  Civelek, Seda\\
  \texttt{e2237147@ceng.metu.edu.tr}
}
\title{CENG 384 - Signals and Systems for Computer Engineers \\
Spring 2020 \\
Written Assignment 2}
\begin{document}
\maketitle



\noindent\rule{19cm}{1.2pt}

\begin{enumerate}

\item %write the solution of q1
    \begin{enumerate}
    % Write your solutions in the following items.
    \item %write the solution of q1a
    \begin{enumerate}
        \item  
        The given system is: 
        \begin{center}
            $y[n] = \Sigma^{\infty}_{k=1} x[n-k]$.
        \end{center} 
        Let $x_1[n]$ be an arbitrary input to this system, and let 
        \begin{center}
            $y_1[n] = \Sigma^{\infty}_{k=1}x_1[n-k]$.
        \end{center} 
        Then consider a second input obtained by shifting $x_1[n]$ in time: 
        \begin{center}
            $x_2[n]=x_1[n-n_0]$.
        \end{center} 
        The output corresponding to $x_2[n]$ is: 
        \begin{center}
            $y_2[n] = \Sigma^{\infty}_{k=1} x_2[n-k] = \Sigma^{\infty}_{k=1}x_1[n-n_0-k]$.
        \end{center}
        Now calculate the $y_1[n]$ with time shift $n_0$
        \begin{center}
            $y_1[n-n_0]=\Sigma^{\infty}_{k=1}x[n-n_0-k]$.
        \end{center}
        We see that $y_2[n] = y_1[n-n_0]$. So, the given system is $\bold{time}$ $\bold{invariant}$.\\
        
        
        \item
        The given system is: 
        \begin{center}
            $y[n] = \Sigma^{\infty}_{k=1} x[n-k]$.
        \end{center} 
        For determining the system is linear or not, we consider two arbitrary inputs $x_1[n]$ and $x_2[n]$.
        \begin{center}
            $x_1[n] \to y_1[n] = \Sigma^{\infty}_{k=1} x_1[n-k]$\\
            $x_2[n] \to y_2[n] = \Sigma^{\infty}_{k=1} x_2[n-k]$\\
        \end{center}
        Let $x_3[n]$ be a linear combination of $x_1[n]$ and $x_2[n]$. That is, 
        \begin{center}
            $x_3[n] = ax_1[n] + bx_2[n]$
        \end{center}
        $a$ and $b$ are arbitrary scalars. The output of input $x_3[n]$
        \begin{equation}
            \begin{split}
                y_3[n] &= \Sigma^{\infty}_{k=1}x_3[n - k] \\
                &= \Sigma^{\infty}_{k=1}ax_1[n-k]+bx_2[n-k]   \\
                &= \Sigma^{\infty}_{k=1}ax_1[n-k] +\Sigma^{\infty}_{k=1}bx_2[n-k]  \\     
                &= a \Sigma^{\infty}_{k=1}x_1[n-k] +b \Sigma^{\infty}_{k=1}x_2[n-k]  \\ 
                &= a y_1[n] + by_2[n]
            \end{split}
        \end{equation}
        
        
        \begin{center}
            
        \end{center}
        So, the given system is $\bold{linear}$.\\  
    
        \item
        $y[n]=\sum^{\infty}_{k=1}x[n-k]$ is $\bold{not}$ $\bold{memoryless}$, because $y[n]$ depends on values of $x[\cdot]$ before time instant $n$.\\
        
        \item
        The given system is:
        \begin{center}
            $y[n]=\sum^{\infty}_{k=1}x[n-k]=x[n-1]+x[n-2]+x[n-3]+...$
        \end{center}
        The given system is $\bold{causal}$ because the value of $y[\cdot]$ at any instant $n$ depends on only the previous values of $x[\cdot]$.
        
        \item
        The given system is:
        \begin{center}
            $x[n] \longrightarrow y[n] = \Sigma^{\infty}_{k=1} x[n-k]$
        \end{center}
        The inverse of the given system is:
        \begin{center}
            $y[n] \longrightarrow x[n] = y[n]-y[n-1]$
        \end{center}
        This system is $\bold{invertible}$.
        
        \item
        A stable system produces a bounded output for any given bounded input. However if we try the $u[n]$ as input which is bounded, the output of the system is not bounded for that input. So, system is $\bold{not}$ $\bold{stable}$. 
        
        
    \end{enumerate}
    
    \item %write the solution of q1b
    \begin{enumerate}
        \item 
        The given system is:
        \begin{center}
            $y(t)=tx(2t+3)$
        \end{center}
        Let $x_1(2t+3)$ be an arbitrary input to this system, and let 
        \begin{center}
            $y_1(t) = tx_1(2t+3)$
        \end{center}
        Let $x_2(t)$ is an input obtained by shifting $x_1(2t+3) $ in time:
        \begin{center}
            $x_2(t) = x_1(2(t-t_0)+3)$
       \end{center}
       The output corresponding to $x_2(t)$ is:
       \begin{center}
           $y_2(t) = tx_2(t) = tx_1(2(t-t_0)+3)$.
       \end{center}
       But,
       \begin{center}
            $y_1(t-t_0) = (t-t_0)x_1(2(t-t_0)+3)$.    
       \end{center}
       As a result $y_2(t) \boldsymbol{\neq} y(t-t_0)$. So the system is $\bold{not}$ $\bold{time}$ $\bold{invariant}$.
       
        \item
        $y(t)=tx(2t+3)$ is $\bold{not}$  $\bold{causal}$  because $y(\cdot)$ is depend on future values of $x(\cdot)$.
        
        \item
         $y(t)=tx(2t+3)$ is $\bold{not}$ $\bold{memoryless}$ because $y(\cdot)$ is not just depend on values of $x(\cdot)$ at that time $t$. 
         
        \item
        For determining the system is linear or not, we consider two arbitrary inputs $x_1(t)$ and $x_2(t)$.
        \begin{center}
            $x_1(t) \longrightarrow y_1(t) = tx_1(2t+3)$\\
            $x_2(t) \longrightarrow y_2(t) = tx_2(2t+3)$
        \end{center}
        Let $x_3(t)$ be an linear combination of $x_1(t)$ and $x_2(t)$. That is,
        \begin{center}
            $x_3(t) = ax_1(t) + bx_2(t)$  
        \end{center}
        $a$ and $b$ are arbitrary scalars. We will provide $x_3(t)$ as an input to that system and output will be 
        \begin{equation}
            \begin{split}
                y_3(t)  &= tx_3(2t+3)\\
                        &= t(ax_1(2t+3) + bx_2(2t+3)) \\
                        &= atx_1(2t+3) + btx_2(2t+3) \\
                        &= ay_1(t) + by_2(t)
            \end{split}
        \end{equation}
        So the  system is $\bold{linear}$.
        
        \item
        The system $y(t) = tx(2t+3)$ is $\bold{not}$ $\bold{invertible}$, because $y(t) = 0$, for $t=0$ independent from the value of $x(t)$. So, we can not determine the value of $x(t)$ from the output of the system.
        \item
        For system $y(t) = tx(2t+3)$, a constant input $x(2t+3) = 1$ yields $y(t) = t$, which is unbounded. So the system is $\bold{unstable}$. 
        
        
    \end{enumerate}
    \end{enumerate}


\item %write the solution of q2
    \begin{enumerate}
    % Write your solutions in the following items.
    \item %write the solution of q2a
    $x(t)-5y(t)=y'(t)$\\
    $y'(t)+5y(t)=x(t)$\\
    \item %write the solution of q2b
    We can find $y(t)$ by calculating and adding $particular$ solution and $homogeneous$ solution:
    \begin{center}
        $y(t)=y_h(t)+y_p(t)$
    \end{center}
    For calculating homogeneous solution part $y_h(t)$, we need the calculation of:
    \begin{center}
        $y'(t) + 5y(t) = 0$
    \end{center}
    For that calculation We assume 
    \begin{center}
        $y_h(t)=Ke^{\lambda t}$
    \end{center}
    and plug it into differential equation. So,
    \begin{equation}
    \begin{split}
       y'(t)+5y(t) & = x(t)  \\
        K\lambda e^{\lambda t}+5Ke^{\lambda t} & = 0  \\
        Ke^{\lambda t}(\lambda + 5)& = 0  \\
        \lambda &= -5  \\
    \end{split}
    \end{equation}
    Homogeneous solution is:
    \begin{center}
        $y_h(t)=K_1 e^{-5t} \text{\quad for } t>0$ \\
    \end{center} 
    For particular solution part $y_p$, given input $x(t)$
    \begin{center}
        $y_p(t)=(ae^{-t}+be^{-3t})u(t)$
    \end{center}
    So,
    \begin{equation}
        \begin{split}
             y'(t)+5y(t) = & x(t)\\
    (-ae^{-t}-3be^{-3t}+5ae^{-t}+5be^{-3t})u(t)= &(e^{-t}+e^{-3t})u(t)\\
    4a = &1\\
    2b = &1
        \end{split}
    \end{equation}
    We obtain,
    \begin{center}
            $a=\dfrac{1}{4}$ and $b=\dfrac{1}{2}$\\
    \end{center}
    For getting $y(t)$, we need to add $y_h(t)$ and $y_p(t)$:
    \begin{equation}
        \begin{split}
            y(t) = & y_h(t) + y_p(t)\\
            = & K_1e^{-5t}+\dfrac{1}{4}e^{-t}+\dfrac{1}{2}e^{-3t} \text{\quad for } t>0 \\
        \end{split}
    \end{equation}
    
    System is initially at rest. So, 
    \begin{equation}
        \begin{split}
            y(0)& =0.\\
            K_1+\dfrac{1}{4}+\dfrac{1}{2} & = 0\\
            K_1& =\dfrac{-3}{4}\\
            y(t)& =(\dfrac{-3}{4}e^{-5t}+\dfrac{1}{4}e^{-t}+\dfrac{1}{2}e^{-3t})u(t)\\
        \end{split}
    \end{equation}
    \end{enumerate}

\item %write the solution of q3     
    \begin{enumerate}
    % Write your solutions in the following items.
    \item %write the solution of q3a
    \textbf{}\\
    $x[n]=2\delta [n]+\delta [n+1]$\\
    \begin{figure} [htbp!]
    %\centering
    \begin{tikzpicture}[scale=0.75] 
      \begin{axis}[
          axis lines=middle,
          xlabel={$n$},
          ylabel={$x[n]$},
          xtick={ -2,..., 0,  ..., 2},
          ytick={0, ..., 2},
          ymin=0, ymax=3,
          xmin=-3, xmax=3,
          every axis x label/.style={at={(ticklabel* cs:1.05)}, anchor=west,},
          every axis y label/.style={at={(ticklabel* cs:1.05)}, anchor=south,},
          grid,
        ]
        \addplot [ycomb, black, thick, mark=*] table [x={n}, y={xn}] {q3a.dat};
      \end{axis}
    \end{tikzpicture}
    \end{figure}
    \\
    $h[n]=\delta [n-1] +2 \delta [n+1]$\\
    
    \begin{figure} [htbp!]
    %\centering
    \begin{tikzpicture}[scale=0.75] 
      \begin{axis}[
          axis lines=middle,
          xlabel={$n$},
          ylabel={$h[n]$},
          xtick={ -2,..., 0,  ..., 2},
          ytick={0, ..., 2},
          ymin=0, ymax=3,
          xmin=-3, xmax=3,
          every axis x label/.style={at={(ticklabel* cs:1.05)}, anchor=west,},
          every axis y label/.style={at={(ticklabel* cs:1.05)}, anchor=south,},
          grid,
        ]
        \addplot [ycomb, black, thick, mark=*] table [x={n}, y={xn}] {q3b.dat};
      \end{axis}
    \end{tikzpicture}
    \end{figure}
    $y[n] = x[n]*h[n]=\sum^{\infty}_{k=-\infty}x[k]h[n-k]$\\
    \\
    $y[n]=x[-1]h[n+1]+x[0]h[n]$ , since for all $n \neq -1,0$ $x[n]=0$\\
    $y[n]=h[n+1]+2h[n]$\\
   
    
        \begin{figure} [!htb]
    %\centering
    \begin{tikzpicture}[scale=0.75] 
      \begin{axis}[
          axis lines=middle,
          xlabel={$n$},
          ylabel={$h[n+1]$},
          xtick={ -2,..., 0,  ..., 2},
          ytick={0, ..., 2},
          ymin=0, ymax=3,
          xmin=-3, xmax=3,
          every axis x label/.style={at={(ticklabel* cs:1.05)}, anchor=west,},
          every axis y label/.style={at={(ticklabel* cs:1.05)}, anchor=south,},
          grid,
        ]
        \addplot [ycomb, black, thick, mark=*] table [x={n}, y={xn}] {q3c.dat};
      \end{axis}
    \end{tikzpicture}
    \end{figure}
        \begin{figure} [!htb]
    %\centering
    \begin{tikzpicture}[scale=0.75] 
      \begin{axis}[
          axis lines=middle,
          xlabel={$n$},
          ylabel={$2h[n]$},
          xtick={ -2,..., 0,  ..., 2},
          ytick={0, ..., 4},
          ymin=0, ymax=4,
          xmin=-3, xmax=3,
          every axis x label/.style={at={(ticklabel* cs:1.05)}, anchor=west,},
          every axis y label/.style={at={(ticklabel* cs:1.05)}, anchor=south,},
          grid,
        ]
        \addplot [ycomb, black, thick, mark=*] table [x={n}, y={xn}] {q3e.dat};
      \end{axis}
    \end{tikzpicture}
    \end{figure}
         \begin{figure} [!htb]
    %\centering
    \begin{tikzpicture}[scale=0.75] 
      \begin{axis}[
          axis lines=middle,
          xlabel={$n$},
          ylabel={$y[n]$},
          xtick={ -2,..., 0,  ..., 2},
          ytick={0, ..., 4},
          ymin=0, ymax=4,
          xmin=-3, xmax=3,
          every axis x label/.style={at={(ticklabel* cs:1.05)}, anchor=west,},
          every axis y label/.style={at={(ticklabel* cs:1.05)}, anchor=south,},
          grid,
        ]
        \addplot [ycomb, black, thick, mark=*] table [x={n}, y={xn}] {q3d.dat};
      \end{axis}
    \end{tikzpicture}
    \end{figure}
    
   
    
    \item %write the solution of q3b
    We provide a solution for $y(t)$ below:
    \begin{equation}
    \begin{split}
        x(t) & = u(t-1) + u(t+1)\\
        h(t) & = e^{-t}sin(t)u(t)\\
        y(t) & = \dfrac{dx(t)}{dt}*h(t) \\
             & = \dfrac{d(u(t-1)+u(t+1))}{dt} * h(t)\\
             & = \left( \frac{du(t-1)}{dt} + \frac{du(t+1)}{dt} \right) * h(t)\\
             & = \left(\delta(t-1) + \delta(t+1) \right)*h(t)\\
             & = \left( \delta(t-1) * h(t) \right) +  \left( \delta(t+1) * h(t) \right)\\
             & = h(t-1) + h(t+1) \\
             & = e^{-(t-1)}sin(t-1)u(t-1) + e^{-(t+1)}sin(t+1)u(t+1)
    \end{split}
    \end{equation}
    \end{enumerate}

\item %write the solution of q4
    \begin{enumerate}
    % Write your solutions in the following items.
    \item %write the solution of q4a
    $y(t)=x(t)*h(t)$ \\
    \begin{equation}
        \begin{split}
             y(t)&=\int^{+\infty}_{-\infty}x(\tau)h(t-\tau)d\tau\\
             &=\int^{+\infty}_{-\infty}e^{-\tau}u(\tau)e^{-2(t-\tau)}u(t-\tau)d\tau\\
             &=\int^{+\infty}_{0}e^{-\tau}e^{-2(t-\tau)}u(t-\tau)d\tau\\
             &=\int^{t}_{0}e^{-\tau}e^{-2(t-\tau)}d\tau\\
             &=\int^{t}_{0}e^{-\tau}e^{-2t}e^{2\tau}d\tau\\
             &=e^{-2t}\int^{t}_{0}e^{-\tau}e^{2\tau}d\tau\\
             &=e^{-2t}\int^{t}_{0}e^{\tau}d\tau\\
             &=e^{-2t}\left(e^\tau \right)|^{t}_{0}\\
             &=e^{-2t}\left(e^t - e^0 \right)\\
             &=e^{-2t}\left(e^t - 1 \right)\\
             &=e^{-t}-e^{-2t}\\
        \end{split}
    \end{equation}
    Thus, the convolution of $x(t)$ with $h(t)$:
    \begin{equation}
        \begin{split}
            y(t) &= x(t) * h(t)\\
            y(t) &= 0, \text{\quad for } t < 0\\
            y(t) &= \left( e^{-t}-e^{-2t} \right)u(t)
        \end{split}
    \end{equation}

    \item %write the solution of q4b
    For this question, we need to consider three cases for the values of t:
    \begin{equation}
		y(t) = \begin{cases}  
		        0 & \mbox{if } t < 0 \\ 
				\int_{0}^{t} e^{3(t - \tau)} d\tau & \mbox{if } 0 \leq t \leq 1 \\
				\int_{0}^{1} e^{3(t - \tau)} d\tau & \mbox{if } 1 < t \end{cases}
	\end{equation}
	After evaluation of integrals:
	\begin{equation}
		y(t) = \begin{cases}  
		        0 &\mbox{if } t < 0 \\ 
				\dfrac{1}{3}\left( e^{3t} - 1 \right) & \mbox{if } 0 \leq t \leq 1 \\
				- \dfrac{1}{3} \left(e^{3(t-1)}-e^{3t}\right) & \mbox{if } 1 < t \end{cases}
	\end{equation}
    \end{enumerate}

\item %write the solution of q5
    \begin{enumerate}
    % Write your solutions in the following items.
    \item %write the solution of q5a
    $2y[n+2]-3y[n+1]+y[n]=0$ \\
    The characteristic polynomial of the system is:
    \begin{equation}
        \begin{split}
            D(z)=2z^2-3z+1&=0\\
            (2z-1)(z-1)&=0\\
            z_1= \frac{1}{2} \text{\quad and } z_2 &=1
        \end{split}
    \end{equation}
    $y[n]=c_1\left(\frac{1}{2}\right)^n+c_2$\\
    Calculation of $c1$ and $c2$:
    \begin{equation}
        \begin{split}
            y[0]&=c_1+c_2=1\\
            y[1]&=\frac{c_1}{2}+c_2=0\\
            c_1 &= 2\text{\quad and } c_2 = -1
        \end{split}
    \end{equation}
    
    \text{Overall solution is:} 
   \begin{equation}
       y[n]=2 \left( \frac{1}{2}\right)^n-1\\
   \end{equation} 
    
    \item %write the solution of q5b
    We assume $y(t)=Ke^{\lambda t}$ and derivatives of $y(t)$:\\
    \begin{equation}
        \begin{split}
            y'(t)&=K\lambda e^{\lambda t}\\
    y''(t)&=K \lambda^2 e^{\lambda t}\\
    y^{(3)}(t)&=K \lambda^3 e^{\lambda t}\\
        \end{split}
    \end{equation}
    Put these values into the equation:
    \begin{equation}
        \begin{split}
            K\lambda^3e^{\lambda t}-3K\lambda^2e^{\lambda t}+4K\lambda e^{\lambda t}-2Ke^{\lambda t}&=0\\
            Ke^{\lambda t}(\lambda^3-3\lambda^2+4\lambda-2)&=0\\
            (\lambda-1)(\lambda^2-2\lambda+2)&=0\\
            \lambda_1 =1, \lambda_2=1-j, \lambda_3&=1+j\\
        \end{split}
    \end{equation}
    
    \text{Solution of y(t) and corresponding derivatives are the following:}
    \begin{equation}
        \begin{split}
            y(t)&=K_1e^t+K_2e^t\sin(t)+K_3e^t\cos(t)\\
            y'(t)&=K_1e^t+K_2e^t\sin(t)+K_2e^t\cos(t)+K_3e^t\cos(t)-K_3e^t\sin(t)\\
            y''(t)&=K_1e^t+2K_2e^t\cos(t)-2K_3e^t\sin(t)\\
        \end{split}
    \end{equation}
    
    \text{To find constant coefficients, we try initial conditions:}
    \begin{equation}
        \begin{split}
            y(0)=K_1+K_3&=3\\
            y'(0)=K_1+K_2+K_3&=1\\
            y''(0)=K_1+2K_2&=2\\
            K_1=6, K_2=-2, K_3&=-3 \\
        \end{split}
    \end{equation}
    
    
    \text{Overall solution is:}
    \begin{equation}
        \begin{split}
            y(t)=6e^t-2e^t\sin(t)-3e^t\cos(t)\\
        \end{split}
    \end{equation}
    \end{enumerate}


\item %write the solution of q6
    \begin{enumerate}
    % Write your solutions in the following items.
    \item %write the solution of q6a
    \begin{equation}
        \begin{split}
            w[n]-\dfrac{1}{2}w[n-1]=x[n]\\
        \end{split}
    \end{equation}
    \text{We can express $w[n]$ as following:}\\
    \begin{equation}
            w[n]=x[n]+\dfrac{1}{2}w[n-1]\\
    \end{equation}
    We need an initial condition. So,
    \begin{equation}
    \begin{split}
        x[n]&= A \delta[n]\\
        w[0]&= x[0]+\dfrac{1}{2}w[-1] = A,\\
        w[1]&= x[1]+\dfrac{1}{2}w[0] = \dfrac{1}{2}A,\\
        w[2]&= x[2]+\dfrac{1}{2}w[1] = \left(  \dfrac{1}{2} \right)^2 A\\
        \cdot\\
        \cdot\\
        \cdot\\
        w[n]&=x[n]+\dfrac{1}{2}w[n-1]=\left(\dfrac{1}{2}\right)^n A.
    \end{split}
    \end{equation}
    Setting A = 1, the impulse response for the system is considered as:
    \begin{equation}
        h_0[n] = \left( \dfrac{1}{2}\right)^nu[n]
    \end{equation}
    
    \item %write the solution of q6b
    Overall impulse response of the system is the following:
    \begin{equation}
        \begin{split}
            h[n] &= h_0[n]*h_0[n]\\
            &=\Sigma^{\infty}_{k=-\infty}h_0[k]h_0[n-k]\\
            &=\Sigma^{\infty}_{k=0}h_0[k]h_0[n-k]\\
            &=\Sigma^{n}_{k=0}\left(\dfrac{1}{2}\right)^k\left(\dfrac{1}{2}\right)^{n-k}\\
            &=\Sigma^{n}_{k=0}\left(\dfrac{1}{2}\right)^n\\
            &=(n+1)\left(\dfrac{1}{2}\right)^n
        \end{split}
    \end{equation}

    \item %write the solution of q6c
    \begin{equation}
        \begin{split}
            x[n] &= w[n] -\dfrac{1}{2}w[n-1] \\
            w[n] &= y[n] -\dfrac{1}{2}y[n-1] \\
            w[n-1]&=y[n-1]-\dfrac{1}{2}y[n-2]\\
            x[n] &= y[n] -\dfrac{1}{2}y[n-1] -\dfrac{1}{2} \left(y[n-1]-\dfrac{1}{2}y[n-2] \right)\\
            x[n] &= y[n] -\dfrac{1}{2}y[n-1] -\dfrac{1}{2} y[n-1]+\dfrac{1}{4}y[n-2]\\
            x[n] &= y[n] -y[n-1] +\dfrac{1}{4}y[n-2]\\
        \end{split}
    \end{equation}
    \end{enumerate}

\end{enumerate}
\end{document}

