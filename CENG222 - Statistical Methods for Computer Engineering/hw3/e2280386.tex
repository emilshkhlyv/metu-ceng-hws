\documentclass[12pt]{article}
\usepackage[utf8]{inputenc}
\usepackage{float}
\usepackage{amsmath}


\usepackage[hmargin=3cm,vmargin=6.0cm]{geometry}
\topmargin=-2cm
\addtolength{\textheight}{6.5cm}
\addtolength{\textwidth}{2.0cm}
\setlength{\oddsidemargin}{0.0cm}
\setlength{\evensidemargin}{0.0cm}
\usepackage{indentfirst}
\usepackage{amsfonts}

\begin{document}

\section*{Student Information}

Name : Emil Shikhaliyev\\
ID : 2280386\\

\section*{Answer 1}
\subsection*{a)}
For computing $95\%$  confidence interval on the difference between the means, first we should write available information: \\
\begin{center}
    $\alpha = 0.05$
\end{center}
For people with age 40 and above:
\begin{center}
    $n = 19$, $\Bar{X} = 3.375$, $s_{X} = 0.96$\\
\end{center}
For people under age 40: 
\begin{center}
    $m = 15$, $\Bar{Y} = 2.05$, $s_{Y} = 1.12$\\
\end{center}
I assume population variances are equal: 
\begin{equation}
    \begin{split}
        \sigma^{2}_{X} = \sigma^{2}_Y = \sigma^2
    \end{split}
\end{equation}
So, confidence interval is:
\begin{equation}
    \begin{split}
        \left[ \Bar{X} - \Bar{Y} - t_{\frac{\alpha}{2}} \cdot s_p \sqrt{\dfrac{1}{n}+\dfrac{1}{m}}, \Bar{X} - \Bar{Y} + t_{\frac{\alpha}{2}} \cdot s_p \sqrt{\dfrac{1}{n}+\dfrac{1}{m}} \right]
    \end{split}
\end{equation}
The pooled standard deviation is:
\begin{equation}
    \begin{split}
        s_p &= \sqrt{\dfrac{(n-1)s^{2}_X+(m-1)s^{2}_Y}{n+m-2}} \\
            &= \sqrt{\dfrac{18 \cdot (0.96)^2 + 14 \cdot (1.12)^2}{19+15-2}}\\
            &= \sqrt{\dfrac{16.589 + 17.562}{32}} \\
            &= \sqrt{1.067}\\
            &= 1.033
    \end{split}
\end{equation}
Degrees of freedom:
\begin{equation}
    \begin{split}
        d.f. = n+m-2 = 19+15-2=32        
    \end{split}
\end{equation}
So, we can obtain the critical value from Table A5 on Textbook with $d.f. = 32$:
\begin{center}
    $t_{0.025} = 2.037$
\end{center}
So, 
\begin{equation}
    \begin{split}
        & \left[ \Bar{X} - \Bar{Y} - t_{\frac{\alpha}{2}} \cdot s_p \sqrt{\dfrac{1}{n}+\dfrac{1}{m}}, \Bar{X} - \Bar{Y} + t_{\frac{\alpha}{2}} \cdot s_p \sqrt{\dfrac{1}{n}+\dfrac{1}{m}} \right] \\
        =& \left[ 3.375 - 2.05 - 2.037 \cdot 1.033 \cdot \sqrt{\dfrac{1}{19}+\dfrac{1}{15}}, 3.375 - 2.05 + 2.037 \cdot 1.033 \cdot \sqrt{\dfrac{1}{19}+\dfrac{1}{15}} \right] \\
        =&[0.5982, 2.0518]
    \end{split}
\end{equation}
\subsection*{b)}
For computing $90\%$  confidence interval on the difference between the means, first we should write available information: \\
\begin{center}
    $\alpha = 0.1$
\end{center}
For people with age 40 and above:
\begin{center}
    $n = 19$, $\Bar{X} = 3.375$, $s_{X} = 0.96$\\
\end{center}
For people under age 40: 
\begin{center}
    $m = 15$, $\Bar{Y} = 2.05$, $s_{Y} = 1.12$\\
\end{center}
I assume population variances are equal: 
\begin{equation}
    \begin{split}
        \sigma^{2}_{X} = \sigma^{2}_Y = \sigma^2
    \end{split}
\end{equation}
So, confidence interval is:
\begin{equation}
    \begin{split}
        \left[ \Bar{X} - \Bar{Y} - t_{\frac{\alpha}{2}} \cdot s_p \sqrt{\dfrac{1}{n}+\dfrac{1}{m}}, \Bar{X} - \Bar{Y} + t_{\frac{\alpha}{2}} \cdot s_p \sqrt{\dfrac{1}{n}+\dfrac{1}{m}} \right]
    \end{split}
\end{equation}
The pooled standard deviation is:
\begin{equation}
    \begin{split}
        s_p &= \sqrt{\dfrac{(n-1)s^{2}_X+(m-1)s^{2}_Y}{n+m-2}} \\
            &= \sqrt{\dfrac{18 \cdot (0.96)^2 + 14 \cdot (1.12)^2}{19+15-2}}\\
            &= \sqrt{\dfrac{16.589 + 17.562}{32}} \\
            &= \sqrt{1.067}\\
            &= 1.033
    \end{split}
\end{equation}
Degrees of freedom:
\begin{equation}
    \begin{split}
        d.f. = n+m-2 = 19+15-2=32        
    \end{split}
\end{equation}
So, we can obtain the critical value from Table A5 on Textbook with $d.f. = 32$:
\begin{center}
    $t_{0.05} = 1.694$
\end{center}
So, 
\begin{equation}
    \begin{split}
        & \left[ \Bar{X} - \Bar{Y} - t_{\frac{\alpha}{2}} \cdot s_p \sqrt{\dfrac{1}{n}+\dfrac{1}{m}}, \Bar{X} - \Bar{Y} + t_{\frac{\alpha}{2}} \cdot s_p \sqrt{\dfrac{1}{n}+\dfrac{1}{m}} \right] \\
        =& \left[ 3.375 - 2.05 - 1.694 \cdot 1.033 \cdot \sqrt{\dfrac{1}{19}+\dfrac{1}{15}}, 3.375 - 2.05 + 1.694 \cdot 1.033 \cdot \sqrt{\dfrac{1}{19}+\dfrac{1}{15}} \right] \\
        =&[0.7206, 1.9294]
    \end{split}
\end{equation}
\subsection*{c)}
We should test \\
\begin{center}
    $H_0 : \mu _X = 3$\\
    $H_A : \mu_X > 3$    
\end{center}
We have \textbf{one-sided right-tail} alternative.\\
So, $\mu_0 = 3$\\
Available information:
\begin{center}
    $n = 19$, $\Bar{X} = 3.375$, $s = 0.96$
\end{center}
Degrees of freedom and $\alpha$ is:
\begin{center}
    $d.f. = 18$ and $\alpha = 0.05$
\end{center}
So, 
\begin{equation}
    t_{\alpha} = 1.734
\end{equation}
Computing the T-statistic:
\begin{equation}
    \begin{split}
        t &= \dfrac{\Bar{X} - \mu_0}{s/\sqrt{n}}\\ \\
        &= \dfrac{3.375-3.0}{\dfrac{0.96}{\sqrt{19}}}\\ \\
        &= 1.703 
    \end{split}
\end{equation}
We can accept null hypothesis because $t < t_{\alpha}$. So we cannot say people with age 40 and above supports BREXIT with $95\%$ confidence level.
\section*{Answer 2}
\subsection*{a)}
\begin{equation}
   H_0: \mu = 20    
\end{equation}
\subsection*{b)}
\begin{equation}
    H_A : \mu \neq 20
\end{equation}
\subsection*{c)}
$n = 11$, $\mu_0 = 20$,  $s = 0.07$, $\Bar{X} = 20.07$.\\
Significance level: $\alpha = 0.01$.\\
Degrees of freedom is $n-1 = 11-1 = 10$. \\
We will use \textbf{two-sided alternative}. So,
\begin{equation}
    \begin{split}
        t_{\frac{\alpha}{2}} = 3.169
    \end{split}
\end{equation}
Compute the T-statistic, 
\begin{equation}
    \begin{split}
        t &= \dfrac{\Bar{X} - \mu_0}{s/\sqrt{n}} \\
        &= \dfrac{20.07 - 20.00}{0.07/\sqrt{11}} \\
        &= 3.3166
    \end{split}
\end{equation}
The accepted region is $A = (-3.169, 3.169)$\\
The rejection region is $R = (-\infty, -3.169] \cup [3.169, \infty)$.\\
The null hypothesis rejected because, $t$ is not in accepted region. They should stop the line.
The corresponding graph are uploaded as another visual file.
\section*{Answer 3}
\subsection*{a)}
\begin{equation}
    \mu_X = \mu_Y
\end{equation}
\subsection*{b)}
\begin{equation}
    \mu_X < \mu_Y
\end{equation}
\subsection*{c)}
We have \textbf{one-sided left-tail alternative}.\\
Available information:\\
\begin{center}
    $\alpha = 0.05$
\end{center}
So, $z_{\alpha} = 1.645$.\\
For new painkiller:
\begin{center}
    $n = 68$, $\Bar{X} = 2.8$, $\sigma_X = 1.7$
\end{center}
For current painkiller:
\begin{center}
    $m = 68$, $\Bar{Y} = 3.0$, $\sigma_Y = 1.4$
\end{center}
\begin{equation}
    \begin{split}
        Z &= \dfrac{\Bar{X} - \Bar{Y}}{\sqrt{\dfrac{\sigma^2_X}{n} + \dfrac{\sigma^2_Y}{m}}} \\
        &= \dfrac{2.8-3.0}{\sqrt{\dfrac{(1.7)^2}{68} + \dfrac{(1.4)^2}{68}}} \\
        &= \dfrac{-0.2}{\sqrt{0.0425+0.0288}}\\
        &= \dfrac{-0.2}{0.267}\\
        &= -0.749
    \end{split}
\end{equation}
The accepted region is: $A = (-1.645, + \infty)$ \\
The rejection region is: $R = (-\infty, -1.645]$ \\
The Null hypothesis rejected, because $Z$ is not in accepted region. So, current painkiller is better. \\
The corresponding graph are uploaded as another visual file.
\end{document}
